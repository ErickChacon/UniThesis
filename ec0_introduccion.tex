\chapter*{Introducci�n}
\thispagestyle{empty} %not enumerate the first page
\label{cha:0_intro}

En esta secci�n se tiene que familiarizar al lector con el tema(s) principal del estudio y crear el deseo de que el lector quiera conocer m�s sobre el estudio.~\cite{Gumiaux2003}

Primero redacta de que se tratar� el estudio, porque es importante y prudente. Establece el escenario para lo que viene despu�s, poniendo partes importantes del tema en su propia perspectiva.~\cite{Diggle2007}

S� directo, no tedioso. Dir�jete a una persona inteligente y bien informada pero que no est� profundamente relacionada con el tema.~\cite{Diggle2007}

Aprovecha la introducci�n para proporcionar informaci�n relacionada a las razones de porque el estudio es propuesto, que se conseguiria y los resultados anticipados.~\cite{Sun2012,Gumiaux2003}

\underline{For this template, the reference examples are not related with the text.}
